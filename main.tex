%Este modelo possui as configurações mais comuns para um TCC.
%Na primeira linha são definidas as configurações principais do documento onde:
%- 12pt indica o tamanho da fonte
%- openright indica que todos os capítulos se iniciam nas folhas da direita
%- twoside indica que o texto será impresso frente e verso (o contrário é oneside)
%- a4paper determinar o tamanho do papel (pode ser utilizado letter se for necessário)
%- tcc indica que o trabalho se trata de um tcc e portanto todo os dados pré-textuais 
%possuem essa informação. Também pode ser utilizado o valor projeto para indicar o projeto de tcc.
\documentclass[12pt,openright, twoside, a4paper, tcc]{cct-uenp}
\usepackage[T1]{fontenc}
\usepackage[utf8]{inputenc}
\usepackage{graphicx}
\DeclareGraphicsExtensions{.pdf,.png,.jpg,.ps,.fig,.eps}
\usepackage[usenames,dvipsnames]{xcolor}
\usepackage{epstopdf}
\usepackage{latexsym}
\usepackage{ae}
\usepackage{hyperref}
\usepackage{array}
\usepackage{tipa}
\usepackage{amssymb}
\usepackage{amsfonts}
\usepackage{booktabs}
\usepackage{dsfont}
\usepackage{textcomp}
\usepackage{cmll}
\usepackage{amsmath}
\usepackage{multirow}
\usepackage{multicol}
\usepackage[portuguese,lined,boxed,commentsnumbered,ruled]{algorithm2e}
\usepackage{stmaryrd}
\usepackage{listings}
\usepackage{framed}
\usepackage{lscape}
\usepackage{pdfpages}
\usepackage{tikz}
\usetikzlibrary{shapes,arrows,automata,positioning}



%Interessante para remover a cor dos links
%\hypersetup{
%	colorlinks,
%	linkcolor={black},
%	citecolor={black},
%	urlcolor={black}
%}
\titulo{Leitor de Cartão Resposta}
\tituloingles{Reply Card Reader}
\palavraschave{Latex. Template ABNT. Editoração de texto.}
\palavraschaveingles{Latex. ABNT. Text editoration.}
\autor{Vinicio Rafael Ruaro}
\citacaoautor{Rafael, N. A.}
\data{2021}

\diadefesa{24 de novembro}
\orientador{Prof(a). Dr(a). Wellington Della Mura} % É membro nato e presidente da Banca Examinadora
% \coorientador{Prof(a). Dr(a). Nome do(a) Coorientador(a)} % Pode ou não ser membro da Banca; se for, deve ser incluído como membro a seguir
\membrobancadois{Prof. Dr. Segundo Membro da Banca}
\instmembrobancadois{Universidade/Instituição do Segundo Membro da Banca}
\membrobancatres{Prof. Dr. Terceiro Membro da Banca}
\instmembrobancatres{Universidade/Instituição do Terceiro Membro da Banca}
\membrobancaquatro{Prof. Ms. Quarto Membro da Banca}
\instmembrobancaquatro{Universidade/Instituição do Quarto Membro da Banca}

\makeindex % compila o indice

\begin{document}

% Retira espaço extra obsoleto entre as frases.
\frenchspacing 
% ----------------------------------------------------------
% ELEMENTOS PRÉ-TEXTUAIS
% ----------------------------------------------------------

% Capa (elemento obrigatório)
\imprimircapa 

% Folha de rosto (elemento obrigatório) % (o * indica que haverá a ficha bibliográfica)
%\imprimirfolhaderosto* 
\imprimirfolhaderosto

%ficha catalográfica (elemento obrigatório apenas se for para a biblioteca)
%\input{fichacatalografica.tex] 

% Folha de aprovação (elemento obrigatório)
\imprimirfolhadeaprovacao 
% ou \includepdf{folhadeaprovacao_final.pdf}

%dedicatória, agradecimentos e epígrafe(elemento obrigatório)
% ---
% Dedicatória (elemento opcional)
% ---
\begin{dedicatoria}
   \vspace*{\fill}
   \centering
   \noindent
   \textit{ Este trabalho é dedicado às crianças adultas que,\\
   quando pequenas, sonharam em se tornar cientistas.} \vspace*{\fill}
\end{dedicatoria}
% ---

% ---
% Agradecimentos (elemento opcional, mas fortemente recomendado)
% ---
\begin{agradecimentos}


\end{agradecimentos}
% ---

% ---
% Epígrafe (elemento opcional)
% ---
\begin{epigrafe}
    \vspace*{\fill}
	\begin{flushright}
	   Epígrafe....
	\end{flushright}
\end{epigrafe}
% ---

%resumos (elemento obrigatório)
% ---
% RESUMOS
% ---

% ---
% Resumo em Português (elemento obrigatório)
% ---
\begin{resumo}
 Segundo a \citeonline[3.1-3.2]{NBR6028:2003}, o resumo deve ressaltar o
 objetivo, o método, os resultados e as conclusões do documento. A ordem e a extensão
 destes itens dependem do tipo de resumo (informativo ou indicativo) e do
 tratamento que cada item recebe no documento original. O resumo deve ser
 precedido da referência do documento, com exceção do resumo inserido no
 próprio documento. (\ldots) As palavras-chave devem figurar logo abaixo do
 resumo, antecedidas da expressão Palavras-chave:, separadas entre si por
 ponto e finalizadas também por ponto.
\end{resumo}
% ---

% ---
% Resumo em Inglês (elemento obrigatório)
% ---
% O ambiente Abstract (com A maiúsculo) é definido no estilo dc-uel
\begin{Abstract}
 This is the english abstract. The Abstract in English should be faithful to the
 Resumo in Portuguese, but not a literal translation.
\end{Abstract}
% ---
 

%lista de figuras, tabelas, símbolos, etc. (elemento opcional)
% ---
% Lista de ilustrações (elemento opcional, mas fortemente recomendado)
% ---
\pdfbookmark[0]{\listfigurename}{lof}
\listoffigures*
\cleardoublepage
% ---

% ---
% Lista de tabelas (elemento opcional, mas fortemente recomendado)
% ---
\pdfbookmark[0]{\listtablename}{lot}
\listoftables*
\cleardoublepage
% ---

% ---
% Lista de abreviaturas e siglas (elemento opcional)
% ---
\begin{siglas}
  \item[ABNT] Associação Brasileira de Normas Técnicas
  \item[BNDES] Banco Nacional de Desenvolvimento Econômico e Social
  \item[IBGE] Instituto Nacional de Geografia e Estatística
  \item[IBICT] Instituto Brasileiro de Informação em Ciência e Tecnologia
  \item[NBR] Norma Brasileira
\end{siglas}
% ---

% ---
% Lista de símbolos (elemento opcional)
% ---
% \begin{simbolos}
%   \item[$ \Gamma $] Letra grega Gama
%   \item[$ \Lambda $] Lambda
%   \item[$ \zeta $] Letra grega minúscula zeta
%   \item[$ \in $] Pertence
% \end{simbolos}
% ---
  

%sumário (elemento obrigatório)
\pdfbookmark[0]{\contentsname}{toc} \tableofcontents* \cleardoublepage  

% ----------------------------------------------------------
% ELEMENTOS TEXTUAIS
% ----------------------------------------------------------
\pagestyle{uenp-header} % Configura cabeçalho para apresentar apenas números de página

% ----------------------------------------------------------
% Capitulos
% ----------------------------------------------------------
\chapter{Introdução}

oi

\section{Contextualizaçao}




Com o advento da globalização a quantidade de informações que conseguem ser trocadas de forma fácil e rápida aumentou muito, um  dos motivos e a criação de  diversos meios de comunicação que temos atualmente, sejam rádios ,telefones, ou serviços de internet, jornais  entre outros. Por culpa disto a informação consegue ser disseminada muito mais rapidamente e da mesma forma o e obtida também muito mais facilmente. Tendo em conta isso logo se prejume que todos que consigam ter uma acesso a algun desses meios consiga obter o que de forma fácil e prática.

Com isso em mente, logo se presume que o conhecimento geral da população aumenta, com isso  uma maior parte pessoas consegue ter acesso para procurar uma melhor qualidade de vida

assim com que uma grande parte dela consiga ter meios para procurar uma melhor qualidade de vida fazendo assim que ela tente obter formas de melhorar o seu auto conhecimento ou se auto valorizar mas dessas formas é conseguir entrar em alguma organização de estudos e da faculdade cursos profinalizantes Sendo assm eles têm uma maior procura por meio das pessoas é para entrada desses cursos.Para você entrar nesse curso necessário que você realize uma prova antes escrita verbal entre outras pois não mente organização que fornece esse curso não consegue suportar entre quantidade de vagas ofertadas e a quantidade de pessoas que tentam entrar.

Grande lembro disso Rio corre no sistema de educação brasileiro é o Enem onde que sua nota você consegue utilizar para programas para entrar em faculdades públicas estaduais ou federais de forma gratuita, pegando esse mesmo exemplo do Enem a quantidade média de de pessoas fazendo a prova é muito grande fazendo assim a correção feita por meio de óleo do humanos seda muito difícil pois o corretor pode se enganar e também a grande quantidadeDe cartão de respostas sempre necessário você tá muito grande fazendo assim que se torne muito difícil conseguir corrigir todas as respostas em tempo hábil.

Uma forma de tentar contornar esse problema é a utilização da computação gráfica para leitura desses formulários porém existe algumas dificuldades pois primeiramente o cartão resposta necessita ser transformado em uma imagem digital para que o software consiga ser utilizado e nesta transformações nessa transformação é necessário cuidar quesitos quantidade de luz refletida no cartão resposta possível deformação do mesmo além de coisas como os orientação se ele está um pouco torto e enfim outros fatores por causa disso alguns software espec alguns software especializados foram criados um desse é o otdr ele quanto segue fazer a captura desse cartão resposta e a transformação de resultados muito bem porém é necessário que o cartão resposta real a folha silen uma folha de carbono e para a digitalização do mesmo é necessário um scanner Específico e somado no país seu curso de utilização do software o seu preto seja bastante elevado dificultando assim a utilização dele em testes em empresas menores, por causa disso esse trabalho vida tentar criar um só ter onde que ele diminuir o custo da obtenção dessas imagens ou conseguir trabalhar de uma forma onde com imagens de qualidade inferiores Sem perder a garantia que de de sucesso em leitura daiamagens

\section{Formulação e Escopo do Problema}

\input{2_textual/Introdução/Formulação e Escopo do Problema}

\section{Justificativas}

{"Eu coloquei algo meio vago, possivelmente/provavelmente eu vou mudar ainda de acordo com as reuniões"}


Esse trabalho tenta solucionar a possivel duvida: 

\section{Objetivos}

Esta seção descreve o objetivo geral e os objetivos específicos deste trabalho.

\subsection{Objetivo Geral}

O objetivo do Trabalho e tentar criar uma solução para a correção de cartão resposta de modo que ela seja seme-automatica onde a parte manual seja apenas a digitalização, essa solução deve ser de facil acesso e com um custo processacional e monetario acessivel para as instituições que forem o aplicativo.

\subsection{Objetivos Específicos}

\input{2_textual/Introdução/Objetivos Específicos}

\section{Organização do Trabalho}

\input{2_textual/Introdução/Organização do Trabalho}



\input{2_textual/fundamentacao}

\input{2_textual/metodologia}

\input{2_textual/desenvolvimento}

\input{2_textual/cronograma}

\input{2_textual/resultados}

\input{2_textual/conclusao}

% ----------------------------------------------------------
% ELEMENTOS PÓS-TEXTUAIS

\bibliography{auxiliar/bibliografia} % Referências bibliográficas

%% ----------------------------------------------------------
% Glossário
% ----------------------------------------------------------
%
% Consulte o manual da classe abntex2 para orientações sobre o glossário.
%
%\glossary


%% ----------------------------------------------------------
% Apêndices
% ----------------------------------------------------------
% ---
% Apêndices (elemento opcional)
%
% São textos ou documentos elaborados pelo autor do trabalho a fim complementar
% a sua argumentação.
% ---
\begin{apendicesenv}

% Imprime uma página indicando o início dos apêndices
\partapendices

% ----------------------------------------------------------
\chapter{Teste}
% ----------------------------------------------------------



\end{apendicesenv}
% ---


%% ----------------------------------------------------------
% Anexos (elemento opcional)
%
% São textos ou documentos, não elaborado pelo autor do trabalho que podem servir como
% ilustração, comprovação ou que contribua de forma relevante com o conteúdo já apresentado.
% ----------------------------------------------------------

% ---
% Inicia os anexos
% ---
\begin{anexosenv}

% Imprime uma página indicando o início dos anexos
\partanexos

% ---
\chapter{Teste Dois}
% ---


\end{anexosenv}


%% ----------------------------------------------------------
% Trabalhos publicados pelo autor
%
% ----------------------------------------------------------
\chapter*{Trabalhos Publicados pelo Autor}
\addcontentsline{toc}{chapter}{Trabalhos Publicados pelo Autor}

\noindent
Trabalhos publicados pelo autor durante o programa.

\vspace{12pt}

\begin{enumerate}

\item Jose da silva, autor2 da silva, orientador da silva, \textbf{Título do artigo}, local onde foi
publicado, mês/ano, editora, número de página, isbn, (Qualis CC 2012, xx)

\item Jose da silva, autor2 da silva, orientador da silva, \textbf{Título do artigo}, local onde foi
publicado, mês/ano, editora, número de página, isbn, (Qualis CC 2012, xx)

\item Jose da silva, autor2 da silva, orientador da silva, \textbf{Título do artigo}, local onde foi
publicado, mês/ano, editora, número de página, isbn, (Qualis CC 2012, xx)

\item Jose da silva, autor2 da silva, orientador da silva, \textbf{Título do artigo}, local onde foi
publicado, mês/ano, editora, número de página, isbn, (Qualis CC 2012, xx)

\end{enumerate}


%%---------------------------------------------------------------------
% INDICE REMISSIVO (elemento opcional)
%---------------------------------------------------------------------
% Requer incluir instruções \index{...} no decorrer do texto, para marcar os termos a serem indexados

\printindex

\end{document}
